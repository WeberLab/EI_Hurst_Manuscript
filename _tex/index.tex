% Options for packages loaded elsewhere
\PassOptionsToPackage{unicode}{hyperref}
\PassOptionsToPackage{hyphens}{url}
\PassOptionsToPackage{dvipsnames,svgnames,x11names}{xcolor}
%
\documentclass[
true
]{sn-jnl}

\usepackage{amsmath,amssymb}
\usepackage{iftex}
\ifPDFTeX
  \usepackage[T1]{fontenc}
  \usepackage[utf8]{inputenc}
  \usepackage{textcomp} % provide euro and other symbols
\else % if luatex or xetex
  \usepackage{unicode-math}
  \defaultfontfeatures{Scale=MatchLowercase}
  \defaultfontfeatures[\rmfamily]{Ligatures=TeX,Scale=1}
\fi
\usepackage{lmodern}
\ifPDFTeX\else  
    % xetex/luatex font selection
\fi
% Use upquote if available, for straight quotes in verbatim environments
\IfFileExists{upquote.sty}{\usepackage{upquote}}{}
\IfFileExists{microtype.sty}{% use microtype if available
  \usepackage[]{microtype}
  \UseMicrotypeSet[protrusion]{basicmath} % disable protrusion for tt fonts
}{}
\makeatletter
\@ifundefined{KOMAClassName}{% if non-KOMA class
  \IfFileExists{parskip.sty}{%
    \usepackage{parskip}
  }{% else
    \setlength{\parindent}{0pt}
    \setlength{\parskip}{6pt plus 2pt minus 1pt}}
}{% if KOMA class
  \KOMAoptions{parskip=half}}
\makeatother
\usepackage{xcolor}
\setlength{\emergencystretch}{3em} % prevent overfull lines
\setcounter{secnumdepth}{-\maxdimen} % remove section numbering
% Make \paragraph and \subparagraph free-standing
\makeatletter
\ifx\paragraph\undefined\else
  \let\oldparagraph\paragraph
  \renewcommand{\paragraph}{
    \@ifstar
      \xxxParagraphStar
      \xxxParagraphNoStar
  }
  \newcommand{\xxxParagraphStar}[1]{\oldparagraph*{#1}\mbox{}}
  \newcommand{\xxxParagraphNoStar}[1]{\oldparagraph{#1}\mbox{}}
\fi
\ifx\subparagraph\undefined\else
  \let\oldsubparagraph\subparagraph
  \renewcommand{\subparagraph}{
    \@ifstar
      \xxxSubParagraphStar
      \xxxSubParagraphNoStar
  }
  \newcommand{\xxxSubParagraphStar}[1]{\oldsubparagraph*{#1}\mbox{}}
  \newcommand{\xxxSubParagraphNoStar}[1]{\oldsubparagraph{#1}\mbox{}}
\fi
\makeatother


\providecommand{\tightlist}{%
  \setlength{\itemsep}{0pt}\setlength{\parskip}{0pt}}\usepackage{longtable,booktabs,array}
\usepackage{calc} % for calculating minipage widths
% Correct order of tables after \paragraph or \subparagraph
\usepackage{etoolbox}
\makeatletter
\patchcmd\longtable{\par}{\if@noskipsec\mbox{}\fi\par}{}{}
\makeatother
% Allow footnotes in longtable head/foot
\IfFileExists{footnotehyper.sty}{\usepackage{footnotehyper}}{\usepackage{footnote}}
\makesavenoteenv{longtable}
\usepackage{graphicx}
\makeatletter
\def\maxwidth{\ifdim\Gin@nat@width>\linewidth\linewidth\else\Gin@nat@width\fi}
\def\maxheight{\ifdim\Gin@nat@height>\textheight\textheight\else\Gin@nat@height\fi}
\makeatother
% Scale images if necessary, so that they will not overflow the page
% margins by default, and it is still possible to overwrite the defaults
% using explicit options in \includegraphics[width, height, ...]{}
\setkeys{Gin}{width=\maxwidth,height=\maxheight,keepaspectratio}
% Set default figure placement to htbp
\makeatletter
\def\fps@figure{htbp}
\makeatother

%%%% Standard Packages

\usepackage{graphicx}%
\usepackage{multirow}%
\usepackage{amsmath,amssymb,amsfonts}%
\usepackage{amsthm}%
\usepackage{mathrsfs}%
\usepackage[title]{appendix}%
\usepackage{xcolor}%
\usepackage{textcomp}%
\usepackage{manyfoot}%
\usepackage{booktabs}%
\usepackage{algorithm}%
\usepackage{algorithmicx}%
\usepackage{algpseudocode}%
\usepackage{listings}%

%%%%

\raggedbottom
\makeatletter
\@ifpackageloaded{caption}{}{\usepackage{caption}}
\AtBeginDocument{%
\ifdefined\contentsname
  \renewcommand*\contentsname{Table of contents}
\else
  \newcommand\contentsname{Table of contents}
\fi
\ifdefined\listfigurename
  \renewcommand*\listfigurename{List of Figures}
\else
  \newcommand\listfigurename{List of Figures}
\fi
\ifdefined\listtablename
  \renewcommand*\listtablename{List of Tables}
\else
  \newcommand\listtablename{List of Tables}
\fi
\ifdefined\figurename
  \renewcommand*\figurename{\textbf{Figure}}
\else
  \newcommand\figurename{\textbf{Figure}}
\fi
\ifdefined\tablename
  \renewcommand*\tablename{\textbf{Table}}
\else
  \newcommand\tablename{\textbf{Table}}
\fi
}
\@ifpackageloaded{float}{}{\usepackage{float}}
\floatstyle{ruled}
\@ifundefined{c@chapter}{\newfloat{codelisting}{h}{lop}}{\newfloat{codelisting}{h}{lop}[chapter]}
\floatname{codelisting}{Listing}
\newcommand*\listoflistings{\listof{codelisting}{List of Listings}}
\captionsetup{labelsep=period}
\makeatother
\makeatletter
\makeatother
\makeatletter
\@ifpackageloaded{caption}{}{\usepackage{caption}}
\@ifpackageloaded{subcaption}{}{\usepackage{subcaption}}
\makeatother
\ifLuaTeX
  \usepackage{selnolig}  % disable illegal ligatures
\fi
\usepackage[]{natbib}
\bibliographystyle{plainnat}
\usepackage{bookmark}

\IfFileExists{xurl.sty}{\usepackage{xurl}}{} % add URL line breaks if available
\urlstyle{same} % disable monospaced font for URLs
\hypersetup{
  pdftitle={Does the Brain's E/I Balance Really Shape Long-Range Temporal Correlations?},
  pdfauthor={, and },
  colorlinks=true,
  linkcolor={blue},
  filecolor={Maroon},
  citecolor={Blue},
  urlcolor={Blue},
  pdfcreator={LaTeX via pandoc}}

\title[Does the Brain's E/I Balance Really Shape Long-Range Temporal
Correlations?]{Does the Brain's E/I Balance Really Shape Long-Range
Temporal Correlations?}

% author setup
\author[1]{\fnm{Lydia} \sur{Sochan}}\email{lydiasochan@gmail.com}\author*[1,2,3]{\fnm{Alexander Mark} \sur{Weber}}\email{aweber@bcchr.ca}
% affil setup
\affil[1]{, \orgname{School of Biomedical Engineering, The University of
British Columbia, Vancouver, BC, Canada}}
\affil[2]{, \orgname{BC Children's Hospital Research Institute, The
University of British Columbia, Vancouver, BC, Canada}}
\affil[3]{, \orgname{Pediatrics, The University of British Columbia,
Vancouver, BC, Canada}}

% abstract 


% keywords

\begin{document}
\maketitle

\textsuperscript{1} School of Biomedical Engineering, The University of
British Columbia, Vancouver, BC, Canada\\
\textsuperscript{2} BC Children's Hospital Research Institute, The
University of British Columbia, Vancouver, BC, Canada\\
\textsuperscript{3} Pediatrics, The University of British Columbia,
Vancouver, BC, Canada

\textsuperscript{*} Correspondence:
\href{mailto:aweber@bcchr.ca}{Alexander Mark Weber
\textless{}aweber@bcchr.ca\textgreater{}}

\subsection{Abstract}\label{abstract}

A 3T multimodal MRI study of healthy adults (n=19; 10 female; 21-54
years) was performed to investigate the potential link between fMRI
long-range temporal correlations and excitatory/inhibitory balance. The
study objective was to determine if the Hurst exponent (HE) -- an
estimate of the self-correlation and signal complexity of the
blood-oxygen-level-dependent (BOLD) signal -- is correlated with the
excitatory-inhibitory (E/I) ratio. Findings in this domain have
implications for neurological and neuropsychiatric conditions with
disrupted E/I balance, such as autism spectrum disorder, schizophrenia,
and Alzheimer's disease. From a methodological perspective, HE is also
considerably easier to accurately measure than E/I ratio. If HE can
serve as a proxy for E/I, it may serve as a useful clinical biomarker
for E/I imbalance. Moreover, E/I has been proposed to serve as a control
parameter for brain criticality, which HE is believed to be a measure
of. Thus, understanding if HE and E/I are correlated would serve to
clarify this relationship. The study collected movie-watching and rest
data including fMRI -- which was used to calculate HE -- and magnetic
resonance spectroscopy (MRS) -- which was used to measure inhibitory and
excitatory neurotransmitters -- GABA and glutamate, respectively. HE was
found to increase with movie compared to rest, while E/I did not change
between conditions. HE and E/I were not correlated during either movie
or rest. This study represents the first attempt to investigate this
connection \emph{in vivo} in humans. We conclude that, at 3T and with
our particular methodologies, no association was found.

\section{Introduction}\label{introduction}

Thirty years ago, functional magnetic resonance imaging (fMRI)
profoundly changed the world of MRI by allowing real-time analysis of
pressing neuropsychological questions
\citep{ogawaMagneticResonanceImaging1990, ogawaBrainMagneticResonance1990, stephanShortHistoryCausal2012}.
While initially used to probe task-based responses, researchers have
more recently developed an interest in studying brain function at rest,
known as resting-state fMRI (rs-fMRI)
\citep{decoRestingBrainsNever2013}, i.e.~to understand how brain
dynamics at rest are related to neurological functioning as well as
individual differences. A critical tool in analyzing these dynamics is
the Hurst exponent (HE)
\citep{campbellMonofractalAnalysisFunctional2022}, a measure of
self-similarity derived from the blood-oxygen-dependent (BOLD) signal.
HE estimates the extent to which the BOLD signal displays long-term
memory, where a higher value indicates a self-similar signal with
long-term positive autocorrelations
\citep{campbellMonofractalAnalysisFunctional2022, beggsBeingCriticalCriticality2012}.
Another way of understanding HE is that a signal with high HE is
fractal: similar temporal signal fluctuations are observed, no matter
the time scale \citep{campbellMonofractalAnalysisFunctional2022}.

HE has also emerged as a valuable tool in clinical research, capturing
changes in BOLD signal dynamics across various neuropsychiatric
conditions. In aging populations for instance, HE has been found to be
elevated
\citep{dongHurstExponentAnalysis2018, winkAgeCholinergicEffects2006};
this relationship has also been found in mild cognitive impairment and
Alzheimer's disease
\citep{maximFractionalGaussianNoise2005, longBrainnetomeAtlasBased2018}.
Additionally, changes in HE have been observed in conditions such as
autism, mood disorders, and brain injury
\citep{laiShiftRandomnessBrain2010, donaTemporalFractalAnalysis2017, weiIdentifyingMajorDepressive2013, jingIdentifyingCurrentRemitted2017, donaFractalAnalysisBrain2017}.
These differences suggest HE may reflect changes in global and local
functioning.

Underlying these observations is the critical brain hypothesis, which
posits that the brain operates at a critical point, a state where order
and disorder are perfectly balanced to enable optimal information
processing
\citep{decoRestingBrainsNever2013, beggsBeingCriticalCriticality2012, barangerChaosComplexityEntropy2000, bassettUnderstandingComplexityHuman2011, zimmernWhyBrainCriticality2020, liangExcitationInhibitionBalance2024, poilCriticalStateDynamicsAvalanches2012, lombardiBalanceExcitationInhibition2017, baumgartenCriticalExcitationinhibitionBalance2019, bruiningMeasurementExcitationinhibitionRatio2020, trakoshisIntrinsicExcitationinhibitionImbalance, gaoInferringSynapticExcitation2017, tianTheoreticalFoundationsStudying2022, rubinovNeurobiologicallyRealisticDeterminants2011}.
When operating at a critical point, the brain is maximally sensitive to
external stimuli, and can dynamically transition between ordered and
disordered states to promote efficient cognitive processing
\citep{decoRestingBrainsNever2013, beggsBeingCriticalCriticality2012, tianTheoreticalFoundationsStudying2022, rubinovNeurobiologicallyRealisticDeterminants2011}.Recent
papers suggest the control parameter underlying the brain's ability to
transition between states is the excitatory-inhibitory (E/I) ratio, the
balance of excitatory and inhibitory neural activity, often
operationalized as the ratio of the primary excitatory-to-inhibitory
neurotransmitters, i.e.~glutamate-to-GABA ratio
\citep{liangExcitationInhibitionBalance2024, lombardiBalanceExcitationInhibition2017, baumgartenCriticalExcitationinhibitionBalance2019, bruiningMeasurementExcitationinhibitionRatio2020, trakoshisIntrinsicExcitationinhibitionImbalance, gaoInferringSynapticExcitation2017}.
It is thought that E/I controls criticality by modulating the brain's
signal-to-noise ratio
\citep{liangExcitationInhibitionBalance2024, rubensteinModelAutismIncreased2003}.

Besides the implications to the critical brain hypothesis, understanding
if E/I is related to HE may allow for easier estimation of excitatory
and inhibitory neurotransmitters, as accurate E/I measurement is
technically challenging \citep{ajramContribution1HMagnetic2019}.
Magnetic resonance spectroscopy (MRS) is the only non-invasive method of
measuring the ratio of glutamate (excitatory) to GABA (inhibitory)
\emph{in vivo} \citep{stanleyFunctionalMagneticResonance2018}.
Unfortunately, it has both limited spatial and temporal resolution
\citep{gaoInferringSynapticExcitation2017, ajramContribution1HMagnetic2019, stanleyFunctionalMagneticResonance2018}.
Consequently, if HE could serve as a proxy for E/I, it would be much
easier to estimate E/I in conditions of interest such as autism,
Alzheimer's, and schizophrenia.

There are a handful of studies suggesting a link between HE and E/I,
however they are all either computational models or animal studies
\citep{liangExcitationInhibitionBalance2024, poilCriticalStateDynamicsAvalanches2012, lombardiBalanceExcitationInhibition2017, baumgartenCriticalExcitationinhibitionBalance2019, bruiningMeasurementExcitationinhibitionRatio2020, trakoshisIntrinsicExcitationinhibitionImbalance, gaoInferringSynapticExcitation2017}.
Moreover, their findings are inconsistent, with some reporting positive
linear, negative linear, or U-shaped relationships between the two
variables (see Table~\ref{tbl-lit}). Thus, there is a need for further
study, both to clarify the nature of a potential E/I-Hurst relationship,
and also to confirm if this relationship indeed exists in the human
brain. Therefore, the current study seeks to investigate the potential
E/I-Hurst relationship in vivo, within the visual cortex during
movie-watching and rest.

\begin{longtable}[]{@{}
  >{\raggedright\arraybackslash}p{(\columnwidth - 10\tabcolsep) * \real{0.1585}}
  >{\raggedright\arraybackslash}p{(\columnwidth - 10\tabcolsep) * \real{0.1707}}
  >{\raggedright\arraybackslash}p{(\columnwidth - 10\tabcolsep) * \real{0.1707}}
  >{\raggedright\arraybackslash}p{(\columnwidth - 10\tabcolsep) * \real{0.1585}}
  >{\raggedright\arraybackslash}p{(\columnwidth - 10\tabcolsep) * \real{0.1707}}
  >{\raggedright\arraybackslash}p{(\columnwidth - 10\tabcolsep) * \real{0.1707}}@{}}
\caption{Summary of Methods for Existing E/I-Hurst
Studies}\label{tbl-lit}\tabularnewline
\toprule\noalign{}
\begin{minipage}[b]{\linewidth}\raggedright
Citation
\end{minipage} & \begin{minipage}[b]{\linewidth}\raggedright
Study Type
\end{minipage} & \begin{minipage}[b]{\linewidth}\raggedright
HE Data Type
\end{minipage} & \begin{minipage}[b]{\linewidth}\raggedright
HE Calculation Method
\end{minipage} & \begin{minipage}[b]{\linewidth}\raggedright
E/I Type
\end{minipage} & \begin{minipage}[b]{\linewidth}\raggedright
E/I-Hurst Relationship
\end{minipage} \\
\midrule\noalign{}
\endfirsthead
\toprule\noalign{}
\begin{minipage}[b]{\linewidth}\raggedright
Citation
\end{minipage} & \begin{minipage}[b]{\linewidth}\raggedright
Study Type
\end{minipage} & \begin{minipage}[b]{\linewidth}\raggedright
HE Data Type
\end{minipage} & \begin{minipage}[b]{\linewidth}\raggedright
HE Calculation Method
\end{minipage} & \begin{minipage}[b]{\linewidth}\raggedright
E/I Type
\end{minipage} & \begin{minipage}[b]{\linewidth}\raggedright
E/I-Hurst Relationship
\end{minipage} \\
\midrule\noalign{}
\endhead
\bottomrule\noalign{}
\endlastfoot
Poil et al.~(2012)\citet{poilCriticalStateDynamicsAvalanches2012} &
Computational with in-house simulated model & Neuronal avalanche size &
Detrendend fluctuation analysis (DFA) & Structural: number of E-to-I
neurons & Inverse U \\
Bruining et
al.~(2020)\citet{bruiningMeasurementExcitationinhibitionRatio2020} &
Computational with model by Poil et al.~(2012); modified in-house &
Neuronal oscillation amplitude & DFA & Structural: number of E-to-I
synapses & Inverse U \\
Gao et al.~(2017)\citet{gaoInferringSynapticExcitation2017} &
Computational; in vivo in rats and macaques & Local field potential
(LFP) & PSD & Estimated from LFP & Positive linear \\
Lombardi et al.~(2017)\citet{lombardiBalanceExcitationInhibition2017} &
Computational with in-house model & Neuronal avalanche size & PSD &
Structural: number of E-to-I neurons & Negative linear \\
Trakoshis et
al.~(2020)\citet{trakoshisIntrinsicExcitationinhibitionImbalance} &
Computational with simulated data; in vivo in mice & fMRI BOLD signal &
Wavelet-based maximum likelihood method & E-to-I synaptic conductance &
Positive linear \\
\end{longtable}

\section{Methods}\label{methods}

\subsection{Study Participants}\label{study-participants}

Ethics approval was granted by the Clinical Research Ethics Board at the
University of British Columbia and BC Children's \& Women's Hospital
(H21-02686). Written informed consent was obtained from all
participants. Twenty-six participants were scanned between the ages of
21 and 53.4.

After our analysis and performing quality assurance (see below), seven
participants were removed for having poor quality data, leaving nineteen
final participants, between the ages of 21.3 and 53.4 (mean ± sd: 30.1 ±
8.7 years; 9 males).


  \bibliography{EIHurst.bib}


\end{document}
